\documentclass[a4paper,12pt]{article}
\usepackage[utf8]{inputenc}
\usepackage{amsmath}
\usepackage{listings}
\usepackage{graphicx}
\usepackage{hyperref}

\title{Risoluzione di Sistemi Lineari di Grandi Dimensioni Utilizzando Calcolo Parallelo su Cluster HPC}
\author{Nome Autore}
\date{\today}

\begin{document}

\maketitle

\begin{abstract}
Questo progetto descrive l'implementazione e l'esecuzione di algoritmi paralleli per la risoluzione di sistemi lineari di grandi dimensioni utilizzando il metodo di Gauss-Seidel su un cluster di calcolo ad alte prestazioni (HPC) presso l'Università di Parma (UniPr). Vengono esplorati tre approcci differenti: MPI, OpenMP e CUDA, per sfruttare al meglio le risorse del cluster HPC.
\end{abstract}

\tableofcontents

\section{Introduzione}
La risoluzione di sistemi lineari di grandi dimensioni è una delle problematiche fondamentali nel campo del calcolo scientifico. Questo progetto mira a implementare il metodo di Gauss-Seidel in modo parallelo utilizzando tre differenti tecnologie: MPI per il calcolo distribuito, OpenMP per il calcolo parallelo su CPU e CUDA per il calcolo parallelo su GPU.

\section{Implementazione}

\subsection{Implementazione MPI}
L'implementazione MPI del metodo di Gauss-Seidel prevede l'utilizzo della libreria MPI per la comunicazione tra processi. Di seguito vengono descritti i passaggi principali dell'algoritmo.

\begin{itemize}
    \item \textbf{Inizializzazione MPI}: Viene inizializzato l'ambiente MPI con \texttt{MPI\_Init}, ottenendo il rank del processo corrente e il numero totale di processi.
    \item \textbf{Divisione del lavoro}: La matrice \( A \) e il vettore \( b \) vengono suddivisi in parti uguali tra i processi utilizzando \texttt{MPI\_Scatter}. Ogni processo riceve un sottoinsieme delle righe della matrice e del vettore.
    \item \textbf{Iterazioni di Gauss-Seidel}: Per ogni iterazione, ogni processo calcola i nuovi valori del proprio sottoinsieme del vettore soluzione \( x \). La somma dei prodotti esclusi dalla diagonale viene calcolata e sottratta dal termine noto, poi divisa per l'elemento diagonale.
    \item \textbf{Comunicazione tra processi}: I nuovi valori calcolati dai processi vengono comunicati a tutti gli altri processi utilizzando \texttt{MPI\_Allgather}, in modo che ogni processo abbia una copia aggiornata del vettore soluzione \( x \).
    \item \textbf{Controllo della convergenza}: Viene calcolata la norma del residuo locale per ogni processo e i risultati vengono ridotti a una somma globale utilizzando \texttt{MPI\_Allreduce}. Se la norma globale è al di sotto di una certa soglia di tolleranza, l'iterazione si interrompe.
    \item \textbf{Pulizia e finalizzazione}: Alla fine dell'algoritmo, la memoria allocata viene liberata e l'ambiente MPI viene chiuso con \texttt{MPI\_Finalize}.
\end{itemize}

\subsection{Implementazione OpenMP}
L'implementazione OpenMP sfrutta la parallelizzazione su CPU tramite direttive di compilazione. I passaggi principali sono i seguenti:

\begin{itemize}
    \item \textbf{Inizializzazione dei dati}: La matrice \( A \) e i vettori \( b \) e \( x \) vengono inizializzati.
    \item \textbf{Iterazioni di Gauss-Seidel}: Utilizzando la direttiva \texttt{\#pragma omp parallel for}, il ciclo principale dell'algoritmo viene parallelizzato. Ogni thread calcola i nuovi valori del vettore \( x \) per un sottoinsieme delle righe della matrice.
    \item \textbf{Controllo della convergenza}: La convergenza viene verificata calcolando la norma del residuo e interrompendo l'iterazione se la norma è inferiore a una soglia di tolleranza.
    \item \textbf{Pulizia}: Alla fine dell'algoritmo, la memoria allocata viene liberata.
\end{itemize}

\subsection{Implementazione CUDA}
L'implementazione CUDA sfrutta la parallelizzazione su GPU per accelerare il calcolo. I passaggi principali sono i seguenti:

\begin{itemize}
    \item \textbf{Allocazione e copia dei dati}: La matrice \( A \) e i vettori \( b \) e \( x \) vengono copiati dalla memoria dell'host alla memoria del dispositivo (GPU) utilizzando \texttt{cudaMalloc} e \texttt{cudaMemcpy}.
    \item \textbf{Kernel di Gauss-Seidel}: Un kernel CUDA viene lanciato per calcolare i nuovi valori del vettore \( x \) in parallelo su più thread. Ogni thread calcola il valore per un singolo elemento del vettore.
    \item \textbf{Iterazioni}: Il kernel viene lanciato ripetutamente per il numero massimo di iterazioni o fino al raggiungimento della convergenza.
    \item \textbf{Copia dei risultati}: I risultati vengono copiati dalla memoria del dispositivo alla memoria dell'host utilizzando \texttt{cudaMemcpy}.
    \item \textbf{Pulizia}: La memoria allocata sulla GPU viene liberata con \texttt{cudaFree}.
\end{itemize}

\section{Risultati}
\subsection{Prestazioni}
Le prestazioni delle implementazioni MPI, OpenMP e CUDA sono state valutate in termini di tempo di esecuzione e velocità di convergenza. I risultati mostrano che ciascun approccio ha vantaggi specifici in base alla natura del problema e alla configurazione hardware.

\subsection{Analisi}
Un'analisi dettagliata delle prestazioni rivela che:
\begin{itemize}
    \item L'implementazione MPI è efficiente per problemi molto grandi distribuiti su più nodi.
    \item L'implementazione OpenMP è semplice da implementare e offre buone prestazioni su sistemi con molte CPU.
    \item L'implementazione CUDA fornisce un'accelerazione significativa per problemi che possono essere suddivisi efficacemente tra molti core della GPU.
\end{itemize}

\section{Conclusioni}
Questo progetto ha dimostrato come diverse tecniche di parallelizzazione possono essere applicate per risolvere sistemi lineari di grandi dimensioni utilizzando il metodo di Gauss-Seidel. Ogni approccio (MPI, OpenMP, CUDA) ha mostrato vantaggi specifici e la scelta della tecnologia dipende dalle risorse disponibili e dalle caratteristiche del problema da risolvere.

\begin{thebibliography}{9}
\bibitem{mpi}
    \textit{Using MPI: Portable Parallel Programming with the Message Passing Interface}, William Gropp, Ewing Lusk, Anthony Skjellum.
\bibitem{openmp}
    \textit{Using OpenMP: Portable Shared Memory Parallel Programming}, Barbara Chapman, Gabriele Jost, Ruud van der Pas.
\bibitem{cuda}
    \textit{CUDA Programming: A Developer's Guide to Parallel Computing with GPUs}, Shane Cook.
\end{thebibliography}

\end{document}
